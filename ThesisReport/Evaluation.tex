\chapter{Evaluation}
\label{chap:Evaluation}

In this chapter, we will go through our experimental hypothesis, testing scenarios, experiments conducted on two sample 
stereo matching algorithms, SGBM and ADCensus mentioned in chapter \ref{chap:Introduction}, and the results with our
proposed evlaution system to assess the benefits of using our evaluation model for a 3D AR application over the general-purpose evluation models; 
Middlebury and Kitti Evaluation stereo evaluation.

\section{Stereo Dataset}
It should be noted that the stereo dataset we have used to conduct the experiments on stereo matching algorithms in our system,
are selected from Kitti Stereo Dataset.
In contrary to Middlebury dataset, Kitti Stereo Project provides stereo images and ground truth disparity maps
that are taken from outdoor scenes under real circumstances. This property of sample images makes them more appropriate 
for evaluating the performance of the stereo algorithms in practical AR application, thus better meeting our objectives in this research.
We have selected 52 samples image pairs among Kitti Stereo dataset based on different photometric and visual properties that are important
in stereo vision and an AR application. Some 
of these properties are listed as follows:
\begin{itemize}
\item light and shading; i.e. selecting scenes containing bright, dim, and dark regions
\item Various depth ranges; i.e. selecting scenes that contains near field, medium field and far field objects  
\item Depth discontinuity and Occlusion
\item Well textured and textureless regions
\end{itemize}


\section{Methodoloy}
Before going through the explanation of the experiments conducted to assess our evaluation model, we phrase our main research question in this
study once more here to better justify the selected experiments. 
As mentioned earlier in chapter \ref{chap:Introduction} our main objective is to investigate "whether using 
stereo matching techniques to find the depth map of the 
surrounding environment in an AR application can meet the requirement of a practical AR system". Therefore, our experiments focus 
on assessing those aspects of our evaluation model that assist the designer of an AR system to better answer that question.
As a result, our first attempt towards evaluating our model is to investigate and demonstrate whether the results of the evaluation process 
are properly measured against the corresponding factors in binocular vision and augmented reality framework.
After confirming the aforementioned property, which is the key property in our model, we investigate the effect of our proposed masking 
procedure. We also demosntrate how the evaluation and comparison of methods occur through our model based the specific factors
which have been the focus of this study, by conducting relevant experiments on sample stereo matching algorithms.

\section{Settings}



