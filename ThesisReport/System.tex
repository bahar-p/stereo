\chapter{Design of Evaluation Scheme}

This chapter walks through the steps taken in order to build up our evaluation framework and justification of each decision taken during this design.

\section{Stereo Correspondence Evaluation}

As briefly mentioned in \ref{chap:Introduction}, there are already different evaluation schemes to assess stereo correspondence solutions. Middlebury Stereo Evaluation \cite{mideval} and 
Kitti Stereo Evaluation \cite{kitti} are the most popular evaluation schemes among researchers and scientists in the field. 
Middlebury and Kitti evaluation schemes are both based on defining certain pixel thresholds for the error between the ground truth disparity and the disparity found by the solution. 
They also provide separate masks for depth discontinuities, occluded regions and the mask of all the area in the image for seperate error and outlier calculation.
However, both of these widely used evaluation models take a general approach towards assessing stereo matching algorithms and have not been designed based on any particular application in which 
stereo matching solutions might be used; therefore, they focus more on the fundamental aspects of designing a stereo matching algorithm as a solution per se to \textit{efficiently}
find the \textit{best matches} of corresponding pixels in stereo pairs. 
However, this perspective may raise some questions in the researcher's mind such as:

\begin{enumerate}
\item What is actually an \textit{efficient} solution? On what basis this \textit{efficienly} can be defined?
\item What is a \textit{best match} of corresponding pixels? How can it be defined?
\end{enumerate}

As a matter of fact, these questions has mainly motivated us to study the evaluation scheme of stereo correspondence solutions from a different point of view. 
In this approach, we have taken steps towards evaluating the stereo correspondence solutions based on the application in which they are going to be used, 
thus better defining and tuning the criteria for \textit{efficiency} and 
finding \textit{the best correspondence match} in the evaluation. Since augmented reality applications have more attracted the researchers's attention in the past few years, 
we have decided to build our evaluation scheme based on an outdoor application in this area.

\section{New Evaluation Scheme}

In an augmneted reality system, there are certain factors that would affect the functionality and effectiveness of the system \cite{dras96,liv05}. These factors may correspond to the surrounding 
enivornment, technology and hardware constraints, or human factors which can be different for each individual. Therefore, in the evaluation of an augmented reality system 
which employs stereo vision techniques in order to build a dense depth map of the surrounding environment, these factors should be carefully considered. 
Since humans are an inseparable component of most of AR systems, due to the time constraint and the existence of many factors and issues that need to be thoroughly 
examined while designing and evaluating the functionality and efficiecy of an AR system, we have mainly focused on the human factors, and the requirement of providing a real-time interaction for
the users, which is actually related to technology constraints. 

In order to be able to design an evaluation scheme for assessing stereo matching solutions in an AR application, we need to primarily understand the relation between these two systems and
the aspects in which they would affect each other while building a practical AR system. An overview of different components of AR system is shown in \ref{fig:AR}.
\textbf{FIG of AR}

Concentrating on the requirements of obtaining a real-time interaction between an AR system and the users, and
also certain human factors which would affect both the augmented reality and stereo vision components of the system,
has revealed the necessity of involving them in the evaluation of stereo correspondence solutions used for generation of the depth map in such systems; hence, in order to 
determine whether a stereo correspondence algorithm can meet the requirements of an outdoor AR application we need to design an evaluation scheme which can properly assess 
the \textit{efficiency} of the algorithm in the system and the accuracy of its disparity result based on the important human factors in an binocular vision and an AR system.
These factors are in fact the concepts related to real-time interaction of an AR system mentioned in \ref{chap:Introduction}, and binocular vision, stereopsis, human perception of depth, 
and stereoacuity as thoroughly described in \ref{chap:BinocularVision}.

In other words, we have proposed and desgined an evaluation scheme to determine the suitability of using stereo matching solutions for generating disparity maps of the surrounding 
environment in an augmented reality framework in terms of \textit{accuracy} and \textit{efficiency; i.e. speed}. 
As a result, unlike Middlebury or Kitti evaluation schemes, we label a pixel in the disparity result as an \textit{outlier} if the stereoacuity
of the relative depth error between the ground truth depth and the depth found by the stereo solution is more than the standard stereoacuity thresholds reported for human visual system based
on standard stereo tests. 
Moreover, since we have used the average stereoacuity for different age groups \cite{garn06}, we are able to determine the performance of the algorithm for users 
at different ages; this
makes the evaluation results more reliable and applicable to practical applications of AR since the users of an AR system might be within different age groups.
In terms of evluating the efficiency of the algorithms to investigate whether they meet the requirements of being part of a real-time application, 
we have also integrated the estimation of the average execution time of the algorithm over the input data sets in the evaluation process.

It should be noted that the input datasets we have used for evaluating two sample stereo matching algorithms mentioned in \ref{chap:Intorduction}
to present the results of our evaluation framework and how they should be interpreted, 
are selected from Kitti Stereo Dataset; we have used Kitti Stereo set because it consists of sample stereo images
taken of outdoor scenes under real circumstances.

\subsection{Architecture}
A highlevel architecture of our evaluation scheme is shown in \ref{fig:architecture}.

\textbf{FIG of Architecture}

As it can be seen input data consisting of stereo images, ground truth disparity and calibration data
are first passed to the evluation module along with required parameters and settings.
After the evaluation process is over, various plots can
be generated according to the paramaters and settings passed in. Plots can then be evaluated, and decisions are made on 
whether the stereo corresponence solution is a proper method to be used in the augmented reality application of interest or not for the generation of depth map.
\subsection{Modules}
The evaluation module consists of different components that are dynamically created as part of the process. These components are:

\begin{itemize}
\item Edge masks generated based on input data 
\item Masked ground-truth (occluded or non-occluded based on setting parameters passed in)
\item Full and masked disparity maps generated by the algorithm 
\end{itemize}

The creation of these components is essential depending on the parameters passed in to the main module to successfully conduct 
the evaluation process.

\subsection{Platform, Technology}
The evaluation process have been implemented on a Unix-based platform using C++ programming language as the language used for implementing 
the core functions of the evaluation and Tool Command Language (TCL) for all the scripts.




