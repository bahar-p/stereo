\chapter{Conclusion}
\label{chap:Conclusion}
In this chapter, we succinctly mention our contributions in this study and specify some interesting paths for future 
research and improvement to our proposed system.

\section{Contributions}
Due to the emergence of various applications which combine different techniques in computer vision 
to build a practical system, developing testbeds which are particularly designed for the evaluation of different components in the criteria of the important factors
in the target application is essential.
Nowadays, building practical AR applications is a challenging problem due to the various constraints that these systems normally face. We believe that
addressing these constraints and attempting to find the efficient solutions are propitious research directions.

In this research, we suggest that stereo correspondence methods can be used in outdoor AR systems
as a practical alternative to conventional and inefficient technologies, 3D laser scanner or depth cameras, for obtaining the depth map 
of the surrounding environment, but only if a real-time implementation is used.
This approach, that is, employing stereo correspondence solutions, requires an evaluation scheme which can effectively evaluate the stereo correspondence methods 
while taking the target application into consideration. As a result, the available
evaluation models, Middlebury and Kitti, will not be sufficient for an effective evaluation of the solutions.
Therefore, our main contribution in this study, is proposing an application-oriented evaluation scheme that is designed in the light of the most important
factors in an outdoor AR system. Since humans are the ultimate users of an AR system, we have focused on the relevant factors in the human visual system
that are important for the real-time interaction with the augmented world and the perception of depth of the surrounding environment. We have integrated
specific metrics in our evaluation system which are measured and subsequently evaluated in the framework of an outdoor AR application, thus effectively
analyzing the performance of the solutions in terms of their accuracy and efficiency, that is its execution time, for the target application. These metrics are
the average stereoacuity over distance, the average number of outliers, the average disparity error and the the average execution time. We also suggest that some specific areas 
in a scene are of more importance to AR applications in outdoor environments. Due to the importance of depth discontinuities and occlusion as depth cues to the human visual system,
we define these regions as the depth edges and their surroundings in the scene. Although our experiments did not prove to be sufficient to investigate the validity of this hypothesis,
we would still hypothesize that these regions are worthy of being studied in AR applications.
In addition, the trade-off between the accuracy and the running time of a stereo algorithm can be studied through our system in the framework of an outdoor AR system, thus
better determining the net benefit of certain post processing steps to the target application.

In conclusion, the experimental results in most cases showed the effectiveness of our approach for evaluation of the stereo solutions in outdoor AR applications, which encourages
further research in this particular direction to improve this model in more useful aspects.
Next, we will mention some aspects in which we believe the system can be improved.

\section{Future Work}
This evaluation model can be improved in a few aspects that we will discuss here.
As seen in the experiments, we could not certainly determine the importance of the depth edges in the scene to the outdoor AR application and subsequently their consideration in the evaluation 
of the stereo algorithms. We believe that a solid conclusion can be made by evaluating more stereo matching solutions
within our model and observing the results in the masked regions and the whole image. A better approach to make a more solid hypothesis and design a better 
evluation is through conducting a user study in which user is wearing see-through AR glasses and a synthetic object is placed in particular regions in the scene based on the depth map generated by the stereo correspondence algorithm. The amount of error and its affect on the human perception of depth must then be observed and evaluated in this study. 

Another interesting aspect of improvement is a rigorous study on the effect of other factors that can affect the effectiveness and usability of the outdoor AR system and, therefore, 
are important to be considered in the evaluation of the method that is used to obtain the depth map of the surrounding environment. 
To name some of these factors we can refer to
the resolution of the display devices in the AR system and the effect of contrast and brightness. 
For this type of evaluation, different video streams or stereo images should be captured using devices with different resolutions 
and from different scenes where the effect of shadow and lighting is well depicted. A segmentation algorithm can then be used to 
differentiate specific regions and evaluate the depth results by the algorithm in each area. 
A user study, similar to the one mentioned previously, can also be conducted here to observe and evaluate the effects of the depth results and 
their errors on the HVS in each case.  
A more complete list of the important factors in AR can be
found in the survey on the perceptual issues in augmented reality by Kruijff et al. \cite{kru10}.

There are different post processing techniques in computer vision that can be used to refine the disparity results, such as color segmentation and plane fitting, 
anisotropic diffusion, and common smoothing filters as Gaussian filter. However, most of these techniques can considerably increase the execution time of the algorithm.
A study of different refining methods and their effect on the accuracy and the running time of the algorithm in the framework of a particular application, 
such as an outdoor AR system, is an interesting topic to investigate and can be a valuable asset to many industrial applications.

Another interesting subject for studying is focusing on the evaluation of the existing GPU-based stereo matching techniques in our system.
Investigating their suitability for integration in an outdoor AR system based on their running time, which is expected to be considerably less than many CPU-based solutions, 
and the accuracy of their results in the light of the relevant factors to an outdoor AR system can be an interesting subject for research.

Furthermore, we believe that it will be of specific value to
assess the benefits of our proposed model and its applicability to other applications of augmented reality, such as underwater environments, in which the amount of noise is more significant. 

We certainly encourage the 
interested researchers to investigate these aspects as we believe the increasing development of the hybrid systems in the fields of augmented reality and stereo vision 
require a more systematic way of evaluation to effectively investigate the usability and effectiveness of the system in the target application.



